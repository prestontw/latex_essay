\documentclass[titlepage]{article}
\usepackage{amssymb}
\usepackage{amsfonts}
\usepackage[notes,natbib,isbn=false]{biblatex-chicago}
\usepackage[margin = 1in]{geometry}
\usepackage{setspace}
\usepackage{fancyhdr}
\doublespacing
\def\vnew{v_{new}}
\pagestyle{fancy}
\fancyhf{}
\lhead{Peter Response Paper}
\rhead{Preston Tunnell Wilson \thepage}

\bibliography{bib}
%maybe use \nohyphenation from biblatex package
%texcount does not words in headers, words in citations, nor words in references

\begin{document}

\title{
    Catchy Title
}

\author{
Preston Tunnell Wilson }

\date{
\today \\
Wordcount: 1250}

\maketitle
%500-700 words, do Peter's reforms constitute a revolution?
% are his decrees revolutionary?
Though Peter the Great's decrees and acts were novel and revolutionary,
his rule itself was not a revolution.
Rather, he maintained complete control the entire time -- 
these changes were means of consolidating and guarding power.
Some of these changes benefited the rest of the Russian populace;
however, many of the old systems were still in place.
They might have been tweaked to Peter's using,
but the fact that they remained signifies that Peter's rule was not a revolution
but instead a new mask over the same old Russia.
This is seen in Peter's decrees on culture, military, servants, and his rule.

In terms of culture, Peter had decrees on everything from clothing to beards.
Requiring people of importance to wear German clothing and ride in German saddles
\footcite[110]{decrees}, though rather totalitarian and controlling,
was nothing more than a superficial change.
Citizens were forced to wear German garb,
but this did not change the fact that they were Russian underneath.
However, Peter did use this degrees regulating outward appearance to gain power 
and consolidate Russian government systems.
For example, within \emph{Peter's Decree on Shaving} \footcite[111]{decrees},
Peter declared that peasants must pay a small toll in order to enter and leave the town.
As if peasants did not have enough freedom, now they were basically forced to stay in the same town.
Furthermore, by Peter declaring that merchants can buy peasant villages
in order to build factories \footcite[113]{decrees}, he forced peasants to stay in the same location
and work on the factory.
Additionally, Peter worked to cease the sale of serfs!
It seems like this declaration had little to no effect,
but at least families were sold together instead of separately \footcite[113]{decrees}.
Thus, though Peter enabled easier access to foreign merchants to Russian resources \footcite[116]{decrees},
he ``permanently assigned'' entire villages of peasants to work these factories.
Therefore, though Peter opened up Russia and made it look more modern,
these changes were based on and consolidated the already existing power of serfdom.

One area in which Peter was quite progressive was with regards to the military.
He called it ``the first of the secular matters for the protection of [the] fatherland.''
As such, officials in the military were of a higher rank than those of court officials\footcite{decrees}.
The sons of officials in the military were also granted nobility,
again demonstrating the importance of the army.
Peter also started a standing military \footcite[114]{decrees}
making his rule seem more like a military occupation rather than a monarchy.
Feofan Prokopovich praises this army and how Peter changed what was before it 
from weak and disorderly to ``useful to the fatherland, terrible to the enemy, renowned and glories everywhere'' \footcite[124]{decrees}.
Peter further gained power for the throne by declaring that
the ruler can decide who the next proper heir is \footcite[115]{decrees}.
He mirrored the European ruling class by accepting the role of emperor from the Russian people.

Peter had more of a surface level impact on Russia, though these changes were controversial.
He was said to have revived Russia \footcite[123]{decrees}.
This might be accurate, as the foundation of his country remained basically the same,
while the revived body wore new clothes.

\singlespacing
\printbibliography
\end {document}
